\documentclass[french]{article}
\usepackage[utf8]{inputenc}
\usepackage[T1]{fontenc}

\usepackage{natbib}
\usepackage[vmargin=3cm,left=4cm,right=4cm]{geometry}

\usepackage{babel}

\usepackage[table,xcdraw]{xcolor}

\usepackage{graphicx}
\usepackage{caption} 
\captionsetup{justification=centering}
\usepackage{subcaption}
% \usepackage{hyperref}
\usepackage[hidelinks]{hyperref}

\begin{document}

%###############################################
\begin{titlepage}

\newcommand{\HRule}{\rule{\linewidth}{0.5mm}} % Defines a new command for the horizontal lines, change thickness here

\center % Center everything on the page
 
%----------------------------------------------------------------------------------------
%	Section Titre
%----------------------------------------------------------------------------------------
\HRule \\[0.4cm]
\vspace{1cm}
{ \huge \bfseries Central Statistical Office of Breizhmeiz \\
\vspace{0.5cm}
(CSOB)}\\ % Title of your document
\vspace{1cm}
\HRule \\[1cm]
 
%----------------------------------------------------------------------------------------
%	Section auteur
%----------------------------------------------------------------------------------------
\vspace{1cm}

\Large \today

\vspace{3cm}

\begin{minipage}{0.4\textwidth}
\begin{center}
\Large \textbf{Auteurs :}\\
\vspace{0.5cm}
Sylvia \textsc{Penfeunteun} \\
Loïc \textsc{Coroller}\\
Fabio \textsc{Cassiano}
\end{center}
\end{minipage}

\vspace{5cm}

\begin{figure}[!ht]
    %\hspace*{-0.5cm}
	\includegraphics[height=0.1\columnwidth]{Image/logo/logo_simplon.png}
	\hspace*{0.5cm}
	\includegraphics[height=0.12\columnwidth]{Image/logo/logo_Isen.png}
	\hspace*{0.5cm}
	\includegraphics[height=0.1\columnwidth]{Image/logo/logo_microsoft.jpg}
\end{figure}

\vfill

\end{titlepage}

\newpage

\tableofcontents

\newpage

\section{Introduction}

\textbf{Nom du projet :} Macromania Breizh - Analyse Statistique

\subsection{Motivations}

Le motivations qui ont mené à ce projet sont les suivantes :

\begin{itemize}
    \item Créer une base de données pour stocké des informations de vente de jeux.
    \item Réaliser des analyses statistique.
\end{itemize}

\subsubsection{Le client}

\textit{Macromania Breizh}, est une nouvelle enseigne de vente de jeux vidéo qui a été créée en 2020 dans la région de Bretagne. L’enseigne possède un magasin sur Brest, et souhaite ouvrir de nouveaux magasins dans le reste de la bretagne dans les prochaines années.

\subsubsection{Le problème}

Le client souhaite ouvrir ses premiers magasins en bretagne, et afin d’organiser au mieux l’espace de ces magasins, l’enseigne souhaite mettre en avant les jeux les plus vendus afin d’améliorer ses ventes et attirer de nouveaux clients.

\subsubsection{L'existant}

Actuellement l’enseigne met en avant les dernières sorties de jeux sur les plateformes les plus vendues selon des avis d’experts et une comparaison avec la concurrence.

\subsubsection{Le besoin non satisfait}

Les études effectuées restent approximatives et les chiffres ne permettent pas de représenter objectivement les ventes, puisqu’ on se rend compte  que les jeux vendus ne sont pas forcément ceux prédits.

\subsubsection{Les objectifs}

Le client dispose des données de vente de nombreux jeux sur la période de 1980 à 2020. Ces données regroupent les ventes pour les 3 plus grands marchés, le marché Nord Américain, le marché Européen, et le Japon, on y retrouve également les autres ventes, ainsi que les ventes global. Le client souhaite organiser ces données dans une base de données. Il souhaite également que quelques statistiques soient réalisées sur le jeu de données, afin de prendre connaissance des plateformes les plus vendeuses.

\subsection{Précisions sur le client}

Le magasin se situe dans la ville de Brest dans le Finistère. Il y a beaucoup de concurrence de proximité, avec des enseignes en concurrence directe car similaires ou de plus grandes enseignes spécialisées.

\subsubsection{Marché}

Le marché du jeu vidéo est en pleine croissance depuis de nombreuses années.
L’enjeu commercial est d’optimiser les ventes et réduire le déstockage.

\section{Documentation}

\subsection{Terminologie métier}

La société est un magasin de \textbf{jeux vidéo} qui vend des jeux et des accessoire. Les jeux sont éditer par différents \textbf{editeurs} et sont vendus sur différents \textbf{plateformes}.

\subsection{Profil des utilisateurs finaux}

Dans ce projet nous avons un seul profil utilisateur final, celui qui souhaite mettre en avant les jeux les plus vendus. Il devra avoir des connaissance en SQL pour pouvoir accèder à une base de données, et pouvoir réaliser des ajouts ou des modifications dans cette base de données.

\subsection{Maintenance}

Aucune maintenance de la part de CSOB n'est prévue pour le moment. Le client devra donc se charger de la maintenance de la base de données.

\section{Fonctions à réaliser}

\subsection{Ce que le système doit faire}

Le système doit être une base de données qui contient les données de vente de jeux. Il doit permettre de réaliser des analyses statistiques sur les données.

Le jupyter notebook livré contiendra quelques statistiques sur les données de vente de jeux, que le client pour exploiter pour organiser son magasin.

\subsection{Ce que le système ne doit pas faire}

Le système ne sera pas gérable par le biais d'un site internet, car aucune interface graphique ne sera livré.

\section{Contraintes du système}

\subsection{Contraintes logicielles}

La base de données doit être développer sur MySQL. Les données statistiques seront réalisé en Python, sur un jupyter notebook.

\subsection{Contraintes d'ergonomie}

Un jupyter noteboolook sera livré pour l'utilisateur final, lui permettant de déployer automatiquement la base de données sur sont ordinateur.

\end{document}